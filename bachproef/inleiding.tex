%%=============================================================================
%% Inleiding
%%=============================================================================

\chapter*{Inleiding}
\label{ch:inleiding}

Blockchain is een relatief nieuwe technologie die enorm aan populariteit aan het winnen is de laatste jaren. Net door deze populariteit wordt er vaak geprobeerd die voor allerlei doelen te gebruiken maar de realiteit is dat blockchain niet meer is dan een trage, in opslag inefficiënte, gedistribueerde databank en helemaal niet zo technologisch spectaculair en dus niet onmiddellijk de beste keuze voor veel toepassingen. Waar blockchain wel in uitblinkt is het oplossen van vertrouwensproblemen door zijn onveranderlijk (immutable) karakter en feit dat alle transacties getekend worden waardoor men dus zekerheid heeft over de tegenpartij. 

Een van de gebieden waarbij men met zekerheid kan spreken over een vertrouwensprobleem is het inrichten van verkiezingen. Vaak worden overheden of organisatoren van beschuldigd met stemmen gefraudeerd te hebben. 

In deze paper ga ik specifiek kijken naar de toepassing van blockchain om verkiezingen te organiseren. Eigenschappen zoals onevranderlijkheid en indentiteitsgarantie zijn hierbij cruciaal en, snelheid en opslagruimte minder. Op het eerste zicht lijkt blockchain dus een zeer goede kandidaat om het frauderisico bij verkiezingen weg te werken.

De inleiding moet de lezer net genoeg informatie verschaffen om het onderwerp te begrijpen en in te zien waarom de onderzoeksvraag de moeite waard is om te onderzoeken. In de inleiding ga je literatuurverwijzingen beperken, zodat de tekst vlot leesbaar blijft. Je kan de inleiding verder onderverdelen in secties als dit de tekst verduidelijkt. Zaken die aan bod kunnen komen in de inleiding~\autocite{Pollefliet2011}:

\section{Probleemstelling}
\label{sec:probleemstelling}

Bij fraude in het verkiezingsproces rijzen er een tweetal problemen. Ten eerste is er voor de bevolking een vertrouwensprobleem tegenover de overheid en/of sommige organiserende partijen zoals stemlokaalverantwoordelijken. 
\newline
\newline
Ten tweede kampt de overheid met een imagoprobleem en vooral sociale onrust als er sprake kan zijn van fraude bij de verkiezingen. Deze sociale rust kan vaak verlammend werken voor een heel land en kan in extreme gevallen zelfs leiden tot gewapende conflicten. Een , weliswaar minder extreem voorbeeld hiervan is duidelijk de laatste presidentsverkiezing in de VS waarover nu meer dan één jaar later nog steeds gespeculeerd wordt over er dan niet sprake is van fraude en inmenging. De sociale onrust in de VS nu is bijzonder groot in de vorm van burgerlijk protest, manifestaties, gerechtelijke onderzoeken en soms zelfs staten of stedelijke besturen die bewust voor kiezen om federale eisen te boycotten. Een voorbeeld hiervan is de 'safe heavens' voor illegale immigranten na de aangekondigde uitwijzingscampagne. \newline
\newline
Een overheid die ter goeder trouw handelt en de bevolking ten allen tijden hebben dus steeds belang bij een onbetwistbare en ongemanipuleerde verkiezing. Het biedt de winnende partij een onbetwiste legitimiteit om te besturen en de verliezende partij de gemoedsrust dat ze eerlijk verloren hebben en het resultaat niet hoeven te betwisten. Het zou met andere woorden zorgen voor sociale vrede zorgen en minder conflict. 


\section{Onderzoeksvraag}
\label{sec:onderzoeksvraag}

Wees zo concreet mogelijk bij het formuleren van je onderzoeksvraag. Een onderzoeksvraag is trouwens iets waar nog niemand op dit moment een antwoord heeft (voor zover je kan nagaan). Het opzoeken van bestaande informatie (bv. ``welke tools bestaan er voor deze toepassing?'') is dus geen onderzoeksvraag. Je kan de onderzoeksvraag verder specifiëren in deelvragen. Bv.~als je onderzoek gaat over performantiemetingen, dan 

\section{Onderzoeksdoelstelling}
\label{sec:onderzoeksdoelstelling}

Wat is het beoogde resultaat van je bachelorproef? Wat zijn de criteria voor succes? Beschrijf die zo concreet mogelijk.

\section{Opzet van deze bachelorproef}
\label{sec:opzet-bachelorproef}

% Het is gebruikelijk aan het einde van de inleiding een overzicht te
% geven van de opbouw van de rest van de tekst. Deze sectie bevat al een aanzet
% die je kan aanvullen/aanpassen in functie van je eigen tekst.

De rest van deze bachelorproef is als volgt opgebouwd:

In Hoofdstuk~\ref{ch:stand-van-zaken} wordt een overzicht gegeven van de stand van zaken binnen het onderzoeksdomein, op basis van een literatuurstudie.

In Hoofdstuk~\ref{ch:methodologie} wordt de methodologie toegelicht en worden de gebruikte onderzoekstechnieken besproken om een antwoord te kunnen formuleren op de onderzoeksvragen.

% TODO: Vul hier aan voor je eigen hoofstukken, één of twee zinnen per hoofdstuk

In Hoofdstuk~\ref{ch:conclusie}, tenslotte, wordt de conclusie gegeven en een antwoord geformuleerd op de onderzoeksvragen. Daarbij wordt ook een aanzet gegeven voor toekomstig onderzoek binnen dit domein.

