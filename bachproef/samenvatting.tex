%%=============================================================================
%% Samenvatting
%%=============================================================================

% TODO: De "abstract" of samenvatting is een kernachtige (~ 1 blz. voor een
% thesis) synthese van het document.
%
% Deze aspecten moeten zeker aan bod komen:
% - Context: waarom is dit werk belangrijk?
% - Nood: waarom moest dit onderzocht worden?
% - Taak: wat heb je precies gedaan?
% - Object: wat staat in dit document geschreven?
% - Resultaat: wat was het resultaat?
% - Conclusie: wat is/zijn de belangrijkste conclusie(s)?
% - Perspectief: blijven er nog vragen open die in de toekomst nog kunnen
%    onderzocht worden? Wat is een mogelijk vervolg voor jouw onderzoek?
%
% LET OP! Een samenvatting is GEEN voorwoord!

%%---------- Nederlandse samenvatting -----------------------------------------
%
% TODO: Als je je bachelorproef in het Engels schrijft, moet je eerst een
% Nederlandse samenvatting invoegen. Haal daarvoor onderstaande code uit
% commentaar.
% Wie zijn bachelorproef in het Nederlands schrijft, kan dit negeren, de inhoud
% wordt niet in het document ingevoegd.

\IfLanguageName{english}{%
\selectlanguage{dutch}
\chapter*{Samenvatting}
\lipsum[1-4]
\selectlanguage{english}
}{}

%%---------- Samenvatting -----------------------------------------------------
% De samenvatting in de hoofdtaal van het document

\chapter*{\IfLanguageName{dutch}{Samenvatting}{Abstract}}

% \lipsum[1-4]
Blockchain technologie is momenteel zeer populair en te pas en te onpas wordt geprobeerd die toe te passen in allerlei sectoren. Wel is het zo dat het geen wondermiddel is die alle huidige oplossingen zal vervangen. Algemeen wordt er aangenomen wordt dat de beste use cases voor blockchain technologie deze zijn waar er een hoge graad van fraudegevoeligheid is, waar er toch een aanzienlijk aantal partijen aan deelnemen, waar anonimiteit een belangrijk criterium kan zijn en waar decentralisatie belangrijk is.
Als we naar de beste use cases voor blockchain technologie kijken zien we dat verkiezingen een ideaal kandidaat zouden zijn voor deze innovatieve technologie. Verkiezingen zijn nu eenmaal zeer fraudegevoelig en moeilijk te controleren, de resultaten worden met regelmaat aangevochten en niet enkel in landen die democratie niet hoog in het vaandel dragen maar ook in traditionele bastions van democratie. Ook nemen er heel veel mensen deel aan verkiezingen en deze moeten liefst anoniem kunnen deelnemen om zowel censuur als represailles uit te sluiten en het feit dat het resultaat decentraal opgeslagen wordt zorgt ervoor dat de garantie er is dat je altijd en overal de juiste, niet gemanipuleerde uitslag kan raadplegen.
Er zijn echter een aantal mogelijk uitdagingen die zouden kunnen voor zorgen dat deze technologie toch niet geschikt zou kunnen zijn voor dit toepassingsgebied. Ik kijk vooral naar de ecologische voetafdruk, grootte van de blockchain zelf wetende dat elke node ofwel deelnemer in principe de volledige blockchain moet downloaden, de technische complexiteit voor de gebruiker en het probleem van identiteitsfraude.
Ik zal in mijn bachelorproef beginnen met een opsomming van de belangrijkste bestaande leveranciers van blockchain-technologieën voor voting-systemen en kijken naar hun sterke en zwakke punten.
Ook zal ik het wettelijk kader onderzoeken en kijken in hoever dit toe te passen is binnen de E.U.
De verwachtingen zijn dat er alhoewel het idee op zich zeer populair is, er niet veel kant-en-klare oplossingen op de markt te vinden zijn en die die er zijn mogelijk nog in de kinderschoenen staan.
Als conclusie zal ik een standaard voting-ballot smartcontract opstellen waarin een aantal tekortkomingen van bestaande oplossingen opgelost zullen proberen te worden. De verwachting is dat bestaande oplossingen allemaal technische of praktische tekortkomingen zullen hebben maar dat er wel oplossingen mogelijk zijn voor kiessystemen op grote schaal. Deze oplossingen zullen wellicht niet lang meer op zich laten wachten gezien het groot aantal stakeholders die zich over dit onderwerp buigen op Internationaal niveau.